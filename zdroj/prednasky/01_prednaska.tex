\section{\texorpdfstring{Zavedení základních pojmů}{Lg}}
\vspace{5mm}
\large

% stránka číslo 2 první přednášky

\textbf{$\mathbb{R}^2$} je reálný vektorový prostor dimenze 2. Definujeme v něm \emph{Euklidovskou normu} a \emph{metriku}:
\begin{itemize}
    \item $|z| = \sqrt{x^2+y^2}$, $z = (x,y)\in\mathbb{R}^2$
    \item $\rho(z,w):= |z-w|$, $z,w\in\mathbb{R}^2$
\end{itemize}

\begin{definition}
Prostor $\mathbb{C}$ je prostor $\mathbb{R}^2$, v němž definujeme navíc:
\begin{itemize}
    \item násobení $(x,y).(u,v) = (xu-yv, xv+yu)$
    \item ztotožňujeme $(x,0) \cong$, neboli $\mathbb{R}\subset\mathbb{C}$
    \item značíme $i = (0,1)$
\end{itemize}
\end{definition}

\begin{properties}
Vlastnosti $\mathbb{C}$. Nechť $z = (x,y)\in\mathbb{C}$.
\begin{itemize}
    \item Potom $z = x+iy$ a $(\pm i)^2 = -1$.
    \item Násobení v $\mathbb{C}$ zahrnuje násobení v $\mathbb{R}$ i násobení skalárem v $\mathbb{R}^2$.
\end{itemize}
\end{properties}

% stránka číslo 3 první přednášky

\begin{notation}
Nechť $z = x+iy$, kde $x,y\in\mathbb{R}$. Potom
\begin{itemize}
    \item $\overline{z}:= x-iy$ je \textit{komplexně sdružená část} k $z$,
    \item $Re(z):= x$ je \textit{reálná část} $z$, $Im(z):= y$ je \textit{imaginární část} $z$,
    \item $|z| = \sqrt{x^2+y^2}$ je \textit{modul} nebo \textit{absolutní hodnota} $z$.
\end{itemize}
\end{notation} 

Dále platí
\begin{itemize}
    \item $|z|^2 = z\overline{z}$, $\overline{zw} = \overline{z}.\overline{w}$, $|zw| = |z|.|w|$, $z+\overline{z} = 2.Re(z)$, $z-\overline{z} = 2i.Im(z)$,
    \item $\frac{1}{z} = \frac{\overline{z}}{|z|^2}$, je-li $z\neq 0$,
    \item $\mathbb{C}$ je těleso.
\end{itemize}

Pozor, $\mathbb{C}$ nelze \emph{rozumně} upořádat!
\begin{itemize}
    \item $i>0\implies -1=i^2>0$,
    \item $i<0\implies -1=i^2>0$.
\end{itemize}

% stránka číslo 4 první přednášky

\section{Lineární zobrazení}
\begin{definition}
$\mathbb{R}^2$ je \emph{reálný vektorový prostor} dimenze 2, jeho báze je $\left((1,0)^T, (0,1)^T\right)$. Obecné $\mathbb{R}$\emph{-lineární zobrazení} $L:\mathbb{R}^2\to\mathbb{R}^2$ má tvar 

\begin{equation}\label{eq:rLinearMap}
\begin{pmatrix}
x \\ y
\end{pmatrix}
\longmapsto
\begin{pmatrix}
a & c\\
b & d
\end{pmatrix}
\begin{pmatrix}
x \\ y
\end{pmatrix}
\text{,}
\end{equation}
kde $a,b,c,d\in\mathbb{R}$.

$\mathbb{C}$ je \emph{komplexní vektorový prostor} dimenze 1, jeho báze je $1$. Obecné $\mathbb{C}$\emph{-lineární zobrazení} $L:\mathbb{C}\to\mathbb{C}$ má tvar $Lz = wz, z\in\mathbb{C}$, kde $w\in\mathbb{C}$. Nechť $z = (x+iy)$, $ w=(a+ib)$. 
Potom $$Lz=(a+ib)(x+iy)= (ax-by, bx+ay) = 
\begin{pmatrix}
a & -b\\
b & a
\end{pmatrix}
\begin{pmatrix}
x \\ y
\end{pmatrix}\text{.}$$
\end{definition}

\begin{observation}
$\mathbb{R}$\emph{-lineární zobrazení} \cref{eq:rLinearMap} je $\mathbb{C}$\emph{-lineární}, právě když $d = a$, $ c = -b$.
\end{observation} 

\begin{note}
$\mathbb{C}$\emph{-lineární zobrazení} jsou velmi specifická $\mathbb{R}$\emph{-lineární zobrazení}.
\end{note}

% stránka číslo 5 první přední přednášky
\begin{agreement}
Nebude-li řečeno něco jiného, \emph{funkce} znamená \emph{komplexní funkci komplexní proměnné}. Na $f: \mathbb{C} \to \mathbb{C}$ se můžeme vždy dívat jako na $f: \mathbb{R}^2 \to \mathbb{R}^2$, protože $\mathbb{C}\approx\mathbb{R}^2$.
Nechť $f$ je funkce z $\mathbb{C}$ do $\mathbb{C}$. Spojitost a limita se definuje stejně jako v základním kurzu matematické analýzy.
\end{agreement}

\begin{definition}
Pro $z_0\in\mathbb{C}, \delta>0$ značíme $U(z_0,\delta):= \{z\in\mathbb{C}: |z-z_0|<\delta\}$ a nazýváme ji \emph{okolí} $z_0$. Dále $P(z_0,\delta):= U(z_0,\delta)\setminus\{z_0\}$ nazýváme \emph{prstencové okolí}. Pokud $\delta$ není důležité, budeme často psát jen $U(z_0)$, $P(z_0)$.

Potom definujeme
\begin{itemize}
    \item $\lim_{z\to x_0} {f(z)} = L$, pokud $\forall\varepsilon>0\ \exists\delta>0:z\in P(x_0,\delta)\implies f(z)\in U(L, \varepsilon)$
    \item $f$ je spojitá v $x_0$, pokud $\lim_{x\to x_0}{f(x)} = f(x_0)$.
\end{itemize}
\end{definition} 

% stránka číslo 6 první přednášky

\section{Diferencovatelnost}

\begin{definition}
Funkce $f$ je v $x_0$ $\mathbb{R}$\emph{-diferencovatelná}, pokud existuje $\mathbb{R}$\emph{-lineární zobrazení} \\$L: \mathbb{R}^2\to\mathbb{R}^2$ takové, že
$$\lim_{h\to 0}\frac{f(x_0+h)-f(x_0)-L(h)}{|h|} = 0.$$
\end{definition} 

\begin{note}

Potom $\diff f(x_0):=L$ je tzv. \emph{totální diferenciál} $f$ v $x_0$ a platí, že $$\diff f(x_0)h:=
\renewcommand\arraystretch{1.5}
\begin{pmatrix}
\frac{\partial f_1}{\partial x}(x_0) & \frac{\partial f_1}{\partial y}(x_0)\\
\frac{\partial f_2}{\partial x}(x_0) & \frac{\partial f_2}{\partial y}(x_0)
\end{pmatrix}h\text{, }h\in \mathbb{R}^2\text{,}$$
kde $f(x,y) = (f_1(x,y),f_2(x,y)).$ (Ta matice se nazývá \emph{Jacobiho matice.})
\end{note} 

% stránka 7 první přednášky

\begin{definition}
Řekneme, že funkce $f$ je v $x_0$ $\mathbb{C}$\emph{-diferencovatelná}, pokud existuje konečná limita $$f'(x_0):= \lim_{h \to 0}\frac{f(x_0+h)-f(x_0)}{h}.$$ Číslo $f'(x_0)$ nazýváme \emph{komplexní derivací} $f$ v $x_0$. 
\end{definition}  

\begin{note}
Jako pro reálnou funkci reálné proměnné platí $(f\pm g)'$, $(f.g)'$, $(f/g)'$ a $(f\circ g)'$.
\end{note} 

\begin{example}\mbox{}
\begin{itemize}
    \item $(z^n)' = n.z^{n-1}$, $z\in \mathbb{C}$ a $n\in \mathbb{N}\text{.}$
    \item $f(z) = \overline{z}$ není nikde v $\mathbb{C}$ $\mathbb{C}$\emph{-diferencovatelná}, ale $f(x,y) = (x,-y)$ je všude $\mathbb{R}$\emph{-diferencovatelná}. Skutečně, máme $$\lim_{h\to 0}\frac{f(x_0+h)-f(x_0)}{h} = \lim_{h\to 0}\frac{\ \overline{h}\ }{h}\text{,}$$ avšak poslední limita neexistuje.
\end{itemize}
\end{example}


% stránka číslo 8 první přednášky

\begin{theorem}[Cauchy-Riemannova]\label{CR}
Nechť $f$ je funkce diferencovatelná na okolí $x_0 \in \mathbb{C}$. Pak následující je ekvivalentní:
\begin{enumerate}
    \item Existuje $f'(x_0)$
    \item Existuje $\diff f(x_0)$ a $\diff f(x_0)$ je $ \mathbb{C}$-lineární
    \item Existuje $\diff f(x_0)$ a v $z_0$ platí tvrzení \emph{Cauchy-Riemannových podmínek}.
\end{enumerate}

\emph{Cauchy-Riemannovy podmínky}:
\begin{align*}
\tag{CR}
\label{eqn:CR}
\frac{\partial f_1}{\partial x} &= \frac{\partial f_2}{\partial y}\text{, }\\
\frac{\partial f_1}{\partial y} &= -\frac{\partial f_2}{\partial x}\text{,}    
\end{align*}
kde $f(x,y) = (f_1(x,y),f_2(x,y))$.

\end{theorem}
\begin{proof}
$(2. \iff 3.)$ plyne z pozorování pro lineární zobrazení
\newline
$(1. \iff 2.)$ Z definice $w = f'(z_0)$ znamená, že 

\begin{equation}\label{eq:cr12-1} 
0 = \lim_{h \to 0} {\frac{f(x_0+h)-f(z_0)-wh}{h}}\text{.} 
\end{equation}
Po vynásobení výrazu v limitě $h/|h|$ dostaneme, že
\begin{equation}\label{eq:cr12-2}
0 = \lim_{h \to 0} \frac{f(z_0+h)-f(z_0)-wh}{|h|}\text{,}
\end{equation}

což je ekvivalentní tomu, že $\diff f(z_0)h=wh$, $h\in \mathbb{C}$. Z \cref{eq:cr12-2} plyne \cref{eq:cr12-1} vynásobením $|h|/h$.
\end{proof}

% stránka číslo 9 první přednášky

\begin{note}\mbox{}
\begin{itemize}
    \item Existuje-li $f'(z_0)$, potom $df(z_0)h = f'(z_0)h, h \in \mathbb{C}$ a $f'(z_0) = \frac{\partial f}{\partial x}(z_0)$
    \item Platí, že \cref{eqn:CR}
    $\iff \frac{\partial f}{\partial x}    = -i\frac{\partial f}{\partial y}$
\end{itemize}
\end{note}

\begin{proof}\mbox{}
\begin{itemize}
    \item $\diff f(x_0)1 = \frac{\partial f_1}{\partial x}(x_0) + i\frac{\partial f_2}{\partial x}(x_0) =: \frac{\partial f}{\partial x}(x_0)$
    \item zřejmé
\end{itemize}
\end{proof}

\begin{example}
Nechť $f(z) = \overline{z}$, pak $f'(x,y) = (x,-y)$. Dále 
$$\frac{\partial f_1}{\partial x} = 1\text{, } \frac{\partial f_1}{\partial y} = 0\text{, } \frac{\partial f_2}{\partial x} = 0\text{, } \frac{\partial f_2}{\partial y} = -1\text{.}$$
Máme, že $f\in C^\infty(\mathbb{R}^2)$, ale v žádném $z\in \mathbb{C}$ nesplňuje \cref{eqn:CR}, proto není nikde\\ $\mathbb{C}$\emph{-diferencovatelná}.
\end{example}

% konec první přednášky
