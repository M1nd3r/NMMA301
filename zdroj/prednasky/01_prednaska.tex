\section{\texorpdfstring{Zavedení základních pojmů}{Zavedení základních pojmu}}
\vspace{5mm}
\large

% stránka číslo 2 první přednášky

\textbf{$\Real ^2$} je reálný vektorový prostor dimenze 2. Definujeme v něm \emph{Euklidovskou normu} a \emph{metriku}:
\begin{itemize}
    \item $|z| = \sqrt{x^2+y^2}$, $z = (x,y)\in\Real ^2$
    \item $\rho(z,w):= |z-w|$, $z,w\in\Real ^2$
\end{itemize}

\begin{definition}
Prostor $\Comp  $ je prostor $\Real ^2$, v němž definujeme navíc:
\begin{itemize}
    \item násobení $(x,y)\cdot(u,v) = (xu-yv, xv+yu)$
    \item ztotožňujeme $(x,0) \cong x$, neboli $\Real \subset\Comp  $
    \item značíme $i = (0,1)$
\end{itemize}
\end{definition}

\begin{notation}
Nechť $z = x+iy$, kde $x,y\in\Real $. Potom
\begin{itemize}
    \item $\overline{z}:= x-iy$ je \textit{komplexně sdružené číslo} k $z$,
    \item $\Rel(z):= x$ je \textit{reálná část} $z$, $\Img(z):= y$ je \textit{imaginární část} $z$,
    \item $|z| = \sqrt{x^2+y^2}$ je \textit{modul} nebo \textit{absolutní hodnota} $z$.
\end{itemize}
\end{notation} 

\begin{properties}
Vlastnosti $\Comp  $. Nechť $z = (x,y)\in\Comp  $.
\begin{itemize}
    \item Potom $z = x+iy$ a $(\pm i)^2 = -1$.
    \item Násobení v $\Comp  $ zahrnuje násobení v $\Real $ i násobení skalárem v $\Real ^2$.
    \item $|z|^2 = z\overline{z}$, $\overline{zw} = \overline{z} \cdot 
    \overline{w}$, $|zw| = |z|\cdot|w|$, $z+\overline{z} = 2\cdot \Rel(z)$, $z-\overline{z} = 2i\cdot \Img(z)$,
    \item $\frac{1}{z} = \frac{\overline{z}}{|z|^2}$, je-li $z\neq 0$,
    \item $\Comp  $ je těleso.
\end{itemize}
\end{properties}

% stránka číslo 3 první přednášky

Pozor, $\Comp  $ nelze \emph{rozumně} uspořádat!
\begin{itemize}
    \item $i>0\implies -1=i^2>0$,
    \item $i<0\implies -1=i^2>0$.
\end{itemize}

% stránka číslo 4 první přednášky

\section{Lineární zobrazení}
\begin{definition}
%Je-li
$\Real ^2$ je \emph{reálný vektorový prostor} dimenze 2, jeho báze je $\left\{(1,0)^T, (0,1)^T\right\}$. Obecné $\Real $\emph{-lineární zobrazení} $L:\Real ^2\to\Real ^2$ má tvar 

\begin{equation}\label{eq:rLinearMap}
\begin{pmatrix}
x \\ y
\end{pmatrix}
\longmapsto
\begin{pmatrix}
a & c\\
b & d
\end{pmatrix}
\begin{pmatrix}
x \\ y
\end{pmatrix}
\text{,}
\end{equation}
kde $a,b,c,d\in\Real $.
%Je-li
$\Comp  $ je \emph{komplexní vektorový prostor} dimenze 1, jeho báze je $\{1\}$. Obecné $\Comp  $\emph{-lineární zobrazení} $L:\Comp  \to\Comp  $ má tvar $Lz = wz, z\in\Comp  $, kde $w\in\Comp  $. Nechť $z = (x+iy)$, $ w=(a+ib)$. 
Potom $$Lz=(a+ib)(x+iy)= (ax-by, bx+ay) = 
\begin{pmatrix}
a & -b\\
b & a
\end{pmatrix}
\begin{pmatrix}
x \\ y
\end{pmatrix}\text{.}$$
%Platí díky tomu, že $\Comp   \cong \Real ^2$.
%Všimneme si, že každé $\Comp  $\emph{-lineární zobrazení} je $\Real $\emph{-lineární zobrazení}, ale matice tohoto lineárního zobrazení je speciální viz pozorování níže.
\end{definition}

\begin{observation}
$\Real $\emph{-lineární zobrazení} \cref{eq:rLinearMap} je $\Comp  $\emph{-lineární}, právě když $d = a$, $ c = -b$.
\end{observation} 

\begin{note}
$\Comp  $\emph{-lineární zobrazení} jsou velmi specifická $\Real $\emph{-lineární zobrazení}.
%Tj. komplexních lin. zob je mnohem méně, než těch reálných.
\end{note}

% stránka číslo 5 první přední přednášky
\begin{agreement}
Nebude-li řečeno něco jiného, \emph{funkce} znamená \emph{komplexní funkci komplexní proměnné}. Na $f: \Comp   \to \Comp  $ se můžeme vždy dívat jako na $f: \Real ^2 \to \Real ^2$, protože $\Comp  \approx\Real ^2$.
Nechť $f$ je funkce z $\Comp  $ do $\Comp  $. Spojitost a limita se definuje stejně jako v základním kurzu matematické analýzy.
\end{agreement}
%Následuje rychlé připomenutí definic z MA
\begin{definition}
Pro $z_0\in\Comp  , \delta>0$ značíme $U(z_0,\delta):= \{z\in\Comp  : |z-z_0|<\delta\}$ a nazýváme ji \emph{okolí} $z_0$. Dále $P(z_0,\delta):= U(z_0,\delta)\setminus\{z_0\}$ nazýváme \emph{prstencové okolí}. Pokud $\delta$ není důležité, budeme často psát jen $U(z_0)$, $P(z_0)$.

Potom definujeme
\begin{itemize}
    \item $\underset{{z\to z_0}}{\lim}{f(z)} = L$, pokud $\forall\varepsilon>0\ \exists\delta>0\ \forall{z\in P(z_0,\delta)}: f(z)\in U(L, \varepsilon)$
    \item $f$ je spojitá v $z_0$, pokud $\underset{{z\to z_0}}{\lim}{f(z)} = f(z_0)$.
\end{itemize}
\end{definition} 

% stránka číslo 6 první přednášky

\section{Diferencovatelnost}
%Funkce $f$ je v daném bodě diferencovatelná, pokud se dá v blízkosti toho bodu aproximovat lineárním zobrazením.
%Jelikož máme dva typy lineárních zobrazení (podle toho zda se na tu rovinu díváme jako na $\Real ^2$, nebo jako na $\Comp  $) tak dává smysl abychom měli i dva typy diferencovatelnosti.Jako první si uvedeme reálnou diferencovatelnost (funkci $f$ aproximujeme $\Real $\emph{-lineárním zobrazením}).
\begin{definition}
Funkce $f$ je v $z_0$ $\Real $\emph{-diferencovatelná}, pokud existuje $\Real $\emph{-lineární zobrazení} \\$L: \Real ^2\to\Real ^2$ takové, že
$$\lim_{h\to 0}\frac{f(z_0+h)-f(z_0)-L(h)}{|h|} = 0.$$
\end{definition} 

\begin{note}

Potom $\diff f(z_0):=L$ je tzv. \emph{totální diferenciál} $f$ v $z_0$ a platí, že 
$$\diff f(z_0)h:=
\renewcommand\arraystretch{1.5}
\begin{pmatrix}
\frac{\partial f_1}{\partial x}(z_0) & \frac{\partial f_1}{\partial y}(z_0)\\
\frac{\partial f_2}{\partial x}(z_0) & \frac{\partial f_2}{\partial y}(z_0)
\end{pmatrix}h\text{,\hspace{3mm} }h\in \Real ^2\text{,}$$
kde $f(x,y) = (f_1(x,y), \ f_2(x,y)).$ (Tato matice se nazývá \emph{Jacobiho matice.})
%hodí se nezapomenout že doted jsme mluvili pouze o reálné diferencovatelnosti. Jako druhou si ted zavedeme komplexní diferencovatelnost(funkci $f$ aproximujeme $\Comp  $\emph{-lineárním zobrazením}).
\end{note} 

% stránka 7 první přednášky

\begin{definition}
Řekneme, že funkce $f$ je v $z_0$ $\Comp  $\emph{-diferencovatelná}, pokud existuje konečná limita 
\[f'(z_0):= \lim_{h \to 0}\frac{f(z_0+h)-f(z_0)}{h}.\] 
%, kde $h$ je komplexní číslo blížící se k nule.
Číslo $f'(z_0)$ nazýváme \emph{komplexní derivací} $f$ v $z_0$. 
\end{definition}  

\begin{note}
Jako pro reálnou funkci reálné proměnné platí $(f\pm g)'=f'\pm{g'}$, $(f\cdot g)'=f'g+g'f$, $(f/g)'=\frac{f'g-g'f}{g^2}$ a $(f\circ g)'=(f'\circ{g}) \cdot g'$.
%Protože s komplexními čísly se počítá stejně jako s reálnými (výjimkou mohou být některé nerovnosti).
%Tak i pro komplexní funkce jedné komplexní proměnné platí stejné věty jako pro reálné funkce jedné reálné proměnné. Tedy platí: $(f\pm g)'$, $(f \cdot g)'$, $(f/g)'$ a $(f\circ g)'$.

\end{note} 

\begin{example}\mbox{}
\begin{itemize}
    \item $(z^n)' = n\cdot z^{n-1}$, $z\in \Comp  $ a $n\in \N\text{.}$
    \item $f(z) = \overline{z}$ není nikde v $\Comp  $ $\Comp  $\emph{-diferencovatelná}, ale $f(x,y) = (x,-y)$ je všude $\Real $\emph{-diferencovatelná}. Skutečně, pro $z_0\in\Comp  $ libovolné, máme $$\lim_{h\to 0}\frac{f(z_0+h)-f(z_0)}{h} = \lim_{h\to 0}\frac{\ \overline{h}\ }{h}\text{,}$$ avšak poslední limita neexistuje. 
    %Plyne z toho,že pro $h$ na reálné ose je $\frac{\ \overline{h}\ }{h}=1$ a pro $h$ na imaginární ose je $\frac{\ \overline{h}\ }{h}=-1$
    \end{itemize}
    %Již víme, že reálných lineárních zobrazení je mnohem více než komplexních lineárních zobrazení. Což znamená,  že pro funkci je mnohem těžší býti komplexně diferencovatelná v daném bodě, než-li v něm být reálně diferencovatelná.
\end{example}


% stránka číslo 8 první přednášky

\begin{theorem}[Cauchy-Riemannova]\label{CR}
Nechť $f$ je funkce definovaná na okolí $z_0 \in \Comp  $. Pak následující tvrzení jsou ekvivalentní:
\begin{enumerate}
    \item Existuje $f'(z_0)$ %existuje komplexní derivace v $z_{0}$ tj. funkce $f$ je v $z_{0}$ komplexně diferencovatelná.
    \item Existuje $\diff f(z_0)$ a $\diff f(z_0)$ je $ \Comp  $-lineární %existuje TD v $z_{0}$ a ten TD je komplexně lineární
    \item Existuje $\diff f(z_0)$ a v $z_0$ platí tzv. \emph{Cauchy-Riemannovy podmínky}: % $f$ je v $z_{0}$ reálně diferencovatelná
\end{enumerate}
%C-R podmínky jsou vztahy mezi parciálními derivacemi funkce $f$
%\emph{Cauchy-Riemannovy podmínky}:
\begin{align*}
\tag{CR}
\label{eqn:CR}
\frac{\partial f_1}{\partial x}(z_0) &= \frac{\partial f_2}{\partial y}(z_0)\text{, }\\
\frac{\partial f_1}{\partial y}(z_0) &= -\frac{\partial f_2}{\partial x}(z_0)\text{,}    
\end{align*}
kde $f(x,y) = (f_1(x,y),f_2(x,y))$.

\end{theorem}
\begin{proof}
$(2. \iff 3.)$: Plyne z pozorování pro lineární zobrazení %Totální diferenciál je komplexně lineární zobrazení $\iff$ jeho matice (Jakobiho matice) má ten speciální tvar (čásla na diagonále jsou stejná a na antidiagonále se liší o znamánko), ale úplnčě to samé řákají pro Jakobián (CR) podmínky.
\newline
$(1. \iff 2.)$ Podle definice $w = f'(z_0)$ znamená, že 

\begin{equation}\label{eq:cr12-1} 
0 = \lim_{h \to 0} {\frac{f(z_0+h)-f(z_0)-wh}{h}}\text{.} 
\end{equation}
Po vynásobení výrazu v limitě $h/|h|$ dostaneme, že
\begin{equation}\label{eq:cr12-2}
0 = \lim_{h \to 0} \frac{f(z_0+h)-f(z_0)-wh}{|h|}\text{,}
\end{equation}% Číslování je trochu matoucí, chci si ho potom upravit na (1) a (2) místo (2) a (3) - čistě pro přehlednost.
což je ekvivalentní tomu, že  $\diff f(z_0)h=wh$, $h\in \Comp  $. Z \cref{eq:cr12-2} plyne \cref{eq:cr12-1} vynásobením $|h|/h$.
\end{proof}

% stránka číslo 9 první přednášky

\begin{note}\mbox{}
\begin{itemize}
    \item Existuje-li $f'(z_0)$, potom $df(z_0)h = f'(z_0)h, h \in \Comp  $ a $f'(z_0) = \frac{\partial f}{\partial x}(z_0)$ %Poslední rovnost platí co do hodnoty, nikoliv co do existence. Muže se stát, že všechny reálné parciální derivace existují, ale ta komplexní nikoliv....proto tam máme to existuje-li $f'(z_0)$, potom...
    \item Platí, že \cref{eqn:CR}
    $\iff \frac{\partial f}{\partial x}(z_0)    = -i\frac{\partial f}{\partial y}(z_0)$
\end{itemize}
\end{note}

\begin{proof}\mbox{}
\begin{itemize}
    \item $\diff f(z_0)1 = \frac{\partial f_1}{\partial x}(z_0) + i\frac{\partial f_2}{\partial x}(z_0) =: \frac{\partial f}{\partial x}(z_0)$
    \item zřejmé %je to jen přeformulování (CR)
\end{itemize}
\end{proof}

\begin{example}
Nechť $f(z) = \overline{z}$, pak $f(x,y) = (x,-y)$. %funkce $f$ vyjádřená v reálných souřadnicích
Dále 
$$\frac{\partial f_1}{\partial x} = 1\text{, } \frac{\partial f_1}{\partial y} = 0\text{, } \frac{\partial f_2}{\partial x} = 0\text{, } \frac{\partial f_2}{\partial y} = -1\text{.}$$
Máme, že $f\in C^\infty(\Real ^2)$, ale v žádném $z\in \Comp  $ nesplňuje \cref{eqn:CR}, proto není nikde\\ $\Comp  $\emph{-diferencovatelná}.
\end{example}
%funkce $f$ je hladká na celé rovině v reálném smyslu, ale v žádném bodě roviny nejsou splněny (CR), nebot $\frac{\partial f_1}{\partial x} = 1 \neq -1 =\frac{\partial f_2}{\partial y} $
% konec první přednášky
