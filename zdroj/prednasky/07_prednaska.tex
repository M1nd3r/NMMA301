\begin{definition}
Nechť $f:\Sp\rightarrow\Sp$ a $z_0,L\in \Sp$. Potom $L=\lim_{z\rightarrow z_0}f(z)$, pokud pro každou $\{z_n\}_{n=1}^\infty\subset\Sp$, $z_0\neq z_n$  platí $z_n\rightarrow z_0\Rightarrow f(z_n)\rightarrow L$.
\end{definition}

\begin{note}
Platí:
\begin{enumerate}
    \item 
        $\lim_{z\rightarrow\infty}f(z)=\lim_{z\rightarrow0}f\left(1/z\right)\text{,}$ má-li alespoň jedna strana smysl.
    \item 
        $\lim_{z\rightarrow z_0}f(z)=\infty\iff\lim_{z\rightarrow z_0}1/f(z)=0$.
\end{enumerate}
\end{note}

\begin{theorem}
[Aritmetika limit v $\Sp$] 
Platí:
\begin{align*}
    \lim_{z\rightarrow z_0}\left(f(z)\pm g(z)\right)&=\lim_{z\rightarrow z_0}f(z)\pm\lim_{z\rightarrow z_0}g(z)\text{,}\\
    \lim_{z\rightarrow z_0}\left(f(z)\cdot g(z)\right)&=\lim_{z\rightarrow z_0}f(z)\cdot\lim_{z\rightarrow z_0}g(z)\text{,}\\ 
    \lim_{z\rightarrow z_0}\frac{f(z)}{g(z)}&=\frac{\lim_{z\rightarrow z_0}f(z)}{\lim_{z\rightarrow z_0}g(z)}\text{,}
\end{align*}
mají-li pravé strany smysl, pokud definujeme $\forall a\in \Comp:\ a/\infty=0,\ \forall a\in \Sp\setminus\{0\}:\ a/0=\infty,\ \forall a \in\Comp:\ a\pm\infty=\infty,\ \forall a\in\Sp\setminus\{0\}:\ a\cdot\infty=\infty$.\\
Nedefinujeme: $0/0,\ \infty/\infty,\ \infty\pm\infty,\ 0\cdot\infty$.
\end{theorem}
\begin{example}
\emph{Racionální funkce} lze chápat jako spojité funkce z $\Sp$ do $\Sp$. Skutečně, nechť $R=P/Q$, kde $P,\ Q$ jsou polynomy, $Q\neq0$ a $P,\ Q$ nemají stejné kořeny.
\begin{enumerate}
    \item 
        Nechť $Q(z_0)=0$. Potom $P(z_0)\neq0$ a $\lim_{z\rightarrow z_0}R(z)=\infty$. Položme $R(z_0):=\infty$.
    \item
        Pokud $R\not\equiv0$, potom 
        $$\lim_{z\rightarrow\infty}R(z)
        =\lim_{z\rightarrow\infty}\frac{\overbrace{a_0}^{\neq0}z^n+\cdots+a_n}{\underbrace{b_0}_{\neq0}z^m+\cdots+b_m}
        =\lim_{z\rightarrow\infty}z^{n-m}\Bigg(\frac{a_0+\frac{a_1}{z}+\frac{a_n}{z^n}}{b_0+\frac{b_1}{z}+\frac{b_m}{z^m}}\Bigg)
        =\threepartdef{0}{n<m,}{\frac{a_0}{b_0}}{n=m,}{\infty}{n>m.}$$ 
        Položme $R(\infty):=\lim_{z\rightarrow\infty}R(z)$.
\end{enumerate}
\end{example}

\subsection{Izolované singularity}
\begin{definition}
Nechť $f$ je holomorfní funkce na $P(z_0)$, ale není holomorfní na $U(z_0)$. Potom $f$ má v $z_0$
\begin{enumerate}
    \item \emph{odstranitelnou singularitu}, existuje-li $\lim_{z\rightarrow z_0}f(z)\in\Comp$,
    \item \emph{pól}, je-li $\lim_{z\rightarrow z_0}f(z)=\infty$,
    \item \emph{podstatnou singularitu}, pokud $\lim_{z\rightarrow z_0}f(z)$ neexistuje.
\end{enumerate}
\end{definition}
\begin{example}
\begin{align*}
     \frac{\sin{z}}{z}& \text{ má v 0 odstranitelnou singularitu,}\\
     \frac{1}{z^{10}}& \text{ má v 0 pól,}\\
     e^{1/z}& \text{ má v 0 podstatnou singularitu.}\\
\end{align*}
\end{example}
\begin{theorem}[O odstranitelné singularitě] Nechť f je holomorfní funkce na $P(z_0)$. Následující tvrzení jsou ekvivalentní:
\begin{enumerate}
    \item $z_0$ je odstranitelná singularita $f$,
    \item existuje $r>0$ tak, že $f$ je omezená na $P(z_0,\, r)$,
    \item existuje $F\in \Holo (U(z_0))$ tak, že $F=f$ na $P(z_0)$.
\end{enumerate}
\end{theorem}

%TODO zkontrolovat následující úmluvu, přidat odkaz
\begin{agreement}
Odstranitelná singularita je vždy odstraněna ve smyslu (3). Dodefinujeme $f$ v $z_0$ holomorfně.
\end{agreement}
\begin{proof}
$(1)\Rightarrow(2)$: triviální, $(2)\Rightarrow(3)$: Položme 
$$g(z):=\twopartdef{(z-z_0)^2f(z)}{z\in P(z_0),}{0}{z=z_0.}$$ 
Potom $g\in \Holo(U(z_0))$, protože 
$$g'(z_0)=\lim_{z\rightarrow z_0}\frac{g(z)-g(z_0)}{z-z_0}=\lim_{z\rightarrow z_0}\underbrace{(z-z_0)}_{\rightarrow0}\underbrace{f(z)}_{omez.}=0.$$
Navíc pro každé $z\in U(z_0)$ je 
$$g(z)=\sum_{n=2}^{\infty}a_n(z-z_0)^n=(z-z_0)^2F(z),$$ kde $$F(z)\stackrel{def.}{:=}\sum_{n=2}^{\infty}a_n(z-z_0)^{n-2},\ z\in U(z_0).$$ Zřejmě $F\in\Holo(U(z_0))$ a $F=f$ na $P(z_0).$
$(3)\Rightarrow(1)$: jasné.
\end{proof}
\begin{theorem}[O pólu] Nechť f je holomorfní funkce na $P(z_0)$. Následující tvrzení jsou ekvivalentní:
\begin{enumerate}
    \item $z_0$ je pól $f$,
    \item $h:=\frac{1}{f}$ a $h(z_0):=0$ má v $z_0$ nulový bod násobnosti $p$ pro nějaké $p\in\N$,
    \item existuje $p\in\N$ tak, že $$\lim_{z\rightarrow z_0}(z-z_0)^pf(z)\in\Comp\setminus\{0\}\text{,}$$
    \item existuje $p\in \N$ tak, že $\forall k\in\Z$ $$\lim_{z\rightarrow z_0}(z-z_0)^kf(z)=\threepartdef{\infty}{k<p,}{\in\Comp\setminus\{0\}}{k=p,}{0}{k>p.}$$
\end{enumerate}
Číslo $p$ z $\circled{2.} - \circled{4.}$ je určeno jednoznačně a nazývá se \emph{násobnost pólu} $z_0$ funkce $f$.
\end{theorem}

\begin{note}
Píšeme $f(z)\sim g(z)$ pro $z\rightarrow z_0$, je-li $\lim_{z\rightarrow z_0}\frac{f(z)}{g(z)}\in\Comp\setminus\{0\}$. Potom $\circled{3.}\iff f(z)\sim\frac{1}{(z-z_0)^p}$, pro $z\rightarrow z_0$.
\end{note} 

\begin{proof}
$\circled{1.}\Rightarrow\circled{2.}$ Protože $\lim_{z\rightarrow z_0}f(z)=\infty$, je $\lim_{z\rightarrow z_0}\frac{1}{f(z)}=0$. Po odstranění odstranitelné singularity má $1/f$ v $z_0$ nulový bod konečné násobnosti $p\in\N$.\\
$\circled{2.}\Rightarrow\circled{3.}$ Existuje $r>0$ a $g\in\Holo(U(z_0))$ tak, že $g\neq 0$ na $U(z_0,\,r)$ a $h(z)=(z-z_0)^p g(z),\ z\in U(z_0,\,r)$. Potom 
$$\lim_{z\rightarrow z_0}(z-z_0)^p\underbrace{f(z)}_{=\frac{1}{h(z)}}=\frac{1}{g(z_0)}\in\Comp\setminus\{0\}.$$
$\circled{3.}\Rightarrow\circled{4.}$ Máme 
$$\lim_{z\rightarrow z_0}(z-z_0)^kf(z)=\lim_{z\rightarrow z_0}(z-z_0)^{k-p}\underbrace{(z-z_0)^pf(z)}_{\in\ \Comp\setminus\{0\}}=\threepartdef{\infty}{k<p,}{\in\Comp\setminus\{0\}}{k=p,}{0}{k>p.}$$
$\circled{4.}\Rightarrow\circled{1.}$ Položíme $k=0$.
\end{proof}

\begin{theorem}[Casorati-Weierstrass] Nechť $f$ je holomorfní funkce na $P(z_0)$. Následující tvrzení jsou ekvivalentní:
\begin{enumerate}
    \item $z_0$ je podstatná singularita $f$,
    \item $\forall r>0: \overline{f(P(z_0,\ r))}=\Comp$.
\end{enumerate}
\end{theorem}
\begin{note}[Velká Picardova věta] $\circled{1.}\iff\circled{3.}$
\begin{enumerate}
\setcounter{enumi}{2}
    \item $\forall r>0:\Comp\setminus f(P(z_0,\ r))$ je nejvýše jednobodová [hluboká věta, důkaz nebude].
\end{enumerate}

\begin{example}
$\exp(\Comp\setminus\{0\})=\Comp\setminus\{0\}$, $\exp(1/z)$ má v 0 podstatnou singularitu.
\end{example}
\end{note}
\begin{proof}
$\circled{2.}\Rightarrow\circled{1.}$ Jasné z definice limity.\\
$\neg\ \circled{2.}\Rightarrow\neg\ \circled{1.}$ Předpokládejme, že existuje $r>0$ tak, že $\Comp\setminus\overline{f(P(z_0,\ r))}\neq\emptyset$ a $f\in\Holo(P(z_0,\ r))$. Potom existuje $U(u_0,\ \beta)\subset\Comp\setminus\overline{f(P(z_0,\ r))}$, speciálně máme, že $0<|z-z_0|<r\Rightarrow|f(z)-u_0|\geq\beta$. 
Definujeme 
\begin{equation*} 
    \tag{*}
    \label{eqn:7.18.*}
    g(z):=\frac{1}{f(z)-u_0},\ z\in P(z_0,\ r)\text{.}
\end{equation*} 
Potom je $g$ holomorfní a $|g|\leq\frac{1}{\beta}$ na $P(z_0,\ r)$. Tedy $z_0$ je odstranitelná singularita a existuje $L:=\lim_{z\rightarrow z_0}g(z)\in\Comp$. Potom máme 
$$\lim_{z\rightarrow z_0}f(z)\stackrel{\cref{eqn:7.18.*}}{=}\lim_{z\rightarrow z_0}\left(u_0+\frac{1}{g(z)}\right)=\twopartdef{\infty}{L=0,}{\in\Comp}{L\neq0.}$$
Tedy $f$ má v $z_0$ buď odstranitelnou singularitu anebo pól.
\end{proof}
\subsection{Laurentovy řady}
\begin{definition}
Nechť $\{a_n\}_{n=-\infty}^{+\infty}\subset\Comp$ a $z_0\in\Comp$. Potom 
\begin{equation}
    \underbrace{\sum_{n=-\infty}^{+\infty}a_n(z-z_0)^n}_{(L)}= \underbrace{\sum_{n=1}^{+\infty}a_{-n}(z-z_0)^{-n}}_{(H)} + \underbrace{\sum_{n=0}^{+\infty}a_n(z-z_0)^n}_{(R)}
\end{equation}
je \emph{Laurentova řada} s koeficienty ${a_n}$ a středem $z_0$. Řada $(R)$ je \emph{regulární část} $(L)$ a řada $(H)$ je \emph{hlavní část} $(L)$. Řekneme, že $(L)$ konverguje, pokud obě její části, tj. $(H)$ i $(R)$, konvergují.
\end{definition}

\begin{example}
$$\exp\left(\frac{1}{z}\right)=\sum_{n=0}^\infty\frac{1}{n!z^n}$$
\end{example}
\begin{properties}[L]
$\circled{1.}$ \emph{Konvergence:} Existují \emph{jediná} $R,r\in[0,\, +\infty]$ tak, že 
\begin{enumerate}
    \item
        řada $(R)$ konverguje absolutně a lokálně stejnoměrně na $|z-z_0|<R$ a diverguje na $|z-z_0|>R$, 
    \item
        řada $(H)$ konverguje absolutně a lokálně stejnoměrně na $|z-z_0|>r$ a diverguje na $|z-z_0|<r$.
\end{enumerate}
%TODO přidat obrázky

$\circled{2.}$ \emph{Součet:} Nechť $0\leq r<R\leq+\infty$ (toto ne vždy platí: může se stát, že řada nekonverguje). Položme \emph{mezikruží} $P(z_0,\,r,\, R) :=\{z\in\Comp:\ r<|z-z_0|<R\}$. Označíme-li součet $(L)$ jako $f$, potom na $P(z_0,\, r,\, R)$ je $f$ holomorfní, řadu $(L)$ tam derivujeme "člen po členu", atd.
\end{properties}

\begin{note} Platí
$P(z_0,\,R)=P(z_0,\,0,\,R)$. %a $P(z_0,\,r)=P(z_0,\,r,\,\infty)$
\end{note}

\begin{proof}
$\circled{1.}$ Číslo $R$ je poloměr konvergence mocninné řady $(R)$. Pro $w=\frac{1}{z-z_0}$ je řada $(H)$ rovna mocninné řadě 
\begin{equation}
    \sum_{n=1}^\infty a_{-n}w^n.
    \tag{*}
    \label{eqn:7.24.*}
\end{equation} Číslo $\frac{1}{r}$ je poloměr konvergence \cref{eqn:7.24.*}.\\
$\circled{2.}$ Plyne opět z Weierstrassovy věty.
\end{proof}

\fbox{Cíl} Ukážeme, že $f\in\Holo(P(z_0,\,r,\, R))$, právě když existuje jediné $(L)$, které má na $P(z_0,\, r,\, R)$ součet $f$.
