\begin{example}
Nechť $\varphi:=\psi^t+[-1;\,1]$, kde $\psi^t(t):=e^{it},\ t\in[0,\,\pi]$. Potom $\mathbb{C}\backslash\langle{\varphi}\rangle=G_0\cup{G_\infty}$, kde $G_0$ je omezená komponenta ("vnitřek") a $G_\infty$ je neomezená komponenta ("vnějšek"). Platí
$${ind}_\varphi{s}=
\left\{
\begin{array}{ll}
1, \hspace{3mm} s\in{G_0}\\
0, \hspace{3mm} s\in{G_\infty}.
\end{array}
\right.$$
Položme $\Tilde{\varphi}:=\psi^-\dotplus[1;\,-1]$,
kde $\psi^-(t):=e^{it},\ t\in[-\pi,\,0]$. Potom máme 
$$1={ind}_{\varphi\dotplus\tilde\varphi}{s}={ind}_\varphi{s}+{ind}_{\tilde{\varphi}}{s}.$$
\end{example}

\section{Speciální typy integrálů}
\begin{theorem}
Nechť $R=P/Q$, kde $P,Q$ jsou polynomy, které nemají společné kořeny a platí 
\begin{enumerate}
    \item $Q\neq0$ na $\mathbb{R}$,
    \item $st(Q)\geq st(P)+2$, kde $st(Q)$ je stupeň polynomu $Q$.
\end{enumerate}
Potom
\begin{equation}
    \int_{-\infty}^\infty{R(x)dx}=2i\pi.\sum_{\begin{array}{cc}
         Q(s)=0  \\
         {Im}{s}>0 
    \end{array}}{res_s{R}}.
    \label{eqn:print}
\end{equation}
\end{theorem}
\begin{proof}
Ukažte, že integrál v \cref{eqn:print} konverguje, právě když platí 1., 2. Nechť $r>0$ a $\varphi_r:=\varphi^1_r\dotplus\varphi^2_r$, kde $\varphi^1_r(t):=t\ t\in[-r,r]$ a $\varphi^2_r(t):=re^{it},\ t\in[0,\pi]$. Je-li $r>0$ tak velké, aby uvnitř $\varphi_r$ ležely všechny póly $R$ z horní polrovviny, potom
\begin{equation}
    2i\pi.\sum_{\begin{array}{cc}
         Q(s)=0  \\
         {Im}{s}>0 
    \end{array}}{res_s{R}}
    \overset{(RV)}{=}
    \int_{\varphi_r}{R}=
    \int_{\varphi_r^1}{R} +\int_{\varphi_r^2}{R}.
    \label{eqn:res*}
\end{equation}
Máme 
$$\int_{\varphi_r^1}{R}=
\int_{-r}^r{R}\overset{r\to\infty}{\longrightarrow}
\int_{-\infty}^\infty{R}.$$
Protože $\int_{\varphi_r^2}{R}\to{0}$ pro $r\to{+\infty}$, dostaneme z \cref{eqn:res*} pro $r\to\infty$, že platí \cref{eqn:print}.  Neboť existuje $C>0$, $r_0>0$ tak, že $|R(z)|\leq\frac{C}{r^2}$, je-li $|z|=r\geq{r_0}$. Máme totiž
$$|R(z)|=\left|\frac{a_0 z^n+a_1 z^{n-1}...+a_n}{b_0 z^m+b_1 z^{m-1}+...+b_m}\right|=\frac{1}{|z|^2}|z|^{n-m+2}.\underset{\overset{z\to\infty}{\longrightarrow}\frac{a_0}{b_0}}{\underbrace{\left|\frac{a_0+\frac{a_1}{z}+...+\frac{a_n}{z^n}}{b_0+\frac{b_1}{z}+...+\frac{b_m}{z^m}}\right|}}.$$
Tedy
$$\left|\int_{\varphi^2_r}{R}\right|\leq V(\varphi^2_r)\cdot\max_{\langle{\varphi^2_r}\rangle}{|R|}
\leq r\pi\frac{C}{r^2}\overset{r\to\infty}{\longrightarrow}0.$$
\end{proof}

\begin{example}
$$I=\underset{sudá}{\int_0^\infty{\frac{x^2+1}{x^4+1}}dx}
=\frac{1}{2}\underset{:=R(x)}{\int_{-\infty}^\infty\frac{x^2+1}
{x^4+1}dx}=i\pi\cdot(res_{z_0}R+res_{z_1}R)$$
$$=-\frac{i\pi}{4\sqrt{2}}\left[(1+i)^2-(1-i)^2\right]=-\frac{i\pi}{4\sqrt{2}}2\cdot 2i=\frac{\pi}{\sqrt{2}},$$ protože
$${res}_{z_k}R=\frac{z_k^2+1}{4z^3_k}\cdot\frac{z_k}{z_k}=
-\frac{1}{4}(z_k^2+1)z_k,$$
$${res}_{z_0}R=-\frac{1}{4}(i+1)(1+i)\frac{1}{\sqrt{2}}=-\frac{1}{4\sqrt{2}}(1+i)^2,$$
$${res}_{z_1}R=-\frac{1}{4}(-i+1)(-1+i)\frac{1}{\sqrt{2}}=\frac{1}{4\sqrt{2}}(1-i)^2.$$
\end{example}

\begin{theorem}
Nechť $R=P/Q$, kde $P,Q$ jsou polynomy, které nemají společné kořeny a platí
\begin{enumerate}
    \item $Q\neq 0$ na $\mathbb{R}$,
    \item $st(Q)\geq st(p)+1$.
\end{enumerate}
Nechť $a>0$. Potom
\begin{equation}
    \label{eqn:2vint}
    \int_{-\infty}^\infty{R(x)e^{iax}}dx = 2i\pi\cdot\sum_{\begin{array}{cc}
         Q(s)=0  \\
         {Im}{s}>0 
    \end{array}}{res_s\left({R(z)e^{iaz}}\right)}.
\end{equation}
\end{theorem}
\begin{proof}
Za cvičení:
\begin{itemize}
    \item Dokažte, že Newtonův integrál v \cref{eqn:2vint} konverguje právě když platí 1., 2.
    \item Jak se spočte tento integrál pro $a<0$?
\end{itemize}
Jako v předešlé větě integrujeme podél $\varphi_r$ funkci $R(z)e^{iaz}$ a pošleme $r\to\infty$. Platí, že
\begin{equation}
    \int_{\varphi_r^2}{R(z)e^{iaz}}dz\overset{r\to\infty}{\longrightarrow}0
    \label{eqn:odhint}
\end{equation}
z \textit{Jordanova Lemmatu} (bylo na 5. cvičení), z 2. totiž máme, že $\lim_{z\to\infty}R(z)=0$.
\end{proof}

\begin{note}
Je-li $a<0$, potom \cref{eqn:odhint} obecně neplatí. V tomto případě je nutno integrovat přes dolní půlkružnici.
\end{note}

\begin{example}
Spočteme Fourierovu transformaci $\mathcal{F}$ funkce $f(x):=\frac{1}{x^2+1}$, kde
$$(\mathcal{F}f)(t):=\frac{1}{2\pi}\int_{-\infty}^\infty{f(x)e^{itx}}dx, \ t\in\mathbb{R}$$.
\begin{itemize}
    \item Nechť $t>0$. Potom $(\mathcal{F}f)(t)=i\cdot{res}_i{(f(z)e^{itz})}=i\cdot\frac{e^{-t}}{2i}=\frac{e^{-t}}{2}$.
    \item Nechť $t<0$. Potom $(\mathcal{F}f)(t)=i\cdot{res}_{(-i)}{(f(z)e^{itz})}\cdot(-1)=-i\cdot\frac{e^{-t}}{2\cdot  r(-i)}=\frac{e^{t}}{2}$.
\end{itemize}
\textbf{Lépe:}
$$(\mathcal{F}f)(t)=\frac{1}{2\pi}\int_{-\infty}^\infty\frac{\cos{(tx)}+i\sin{(tx)}}{1+x^2}dx=\frac{e^{-|t|}}{2}, \ t\in\mathbb{R}.$$
\end{example}

\begin{example}
$$\int_0^\infty\frac{x\sin{x}}{x^2+1}dx=\frac{1}{2}{Im}\underset{:=I}{\underbrace{\int_{-\infty}^\infty\frac{x.e^{ix}}{x^2+1}dx}}=\frac{\pi}{2e},$$
protože
$$I=2\pi i\,{res}_{i}{f}=2\pi i \frac{ie^{-1}}{2i}=\frac{\pi i}{e}.$$
\end{example}