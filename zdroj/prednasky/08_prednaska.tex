% přednáška 9. týden - 8. týden byl statní svátek

\subsection{Holomorfní funkce na mezikruží}
\begin{lemma}
Nechť $f$ je holomorfní funkce na $P(z_0,\,r,\,R):=\{z \in \mathbb{C}| r<|z-z_0|<R \}$, kde $0\leq r < R \leq + \infty$. Pro každé $\rho \in (r,\,R)$ označme
\begin{equation}
     \varphi_\rho(t):=z_0+\rho e^{it} \text{, pro } t\in [0,\,2 \pi]
     \tag{$\triangle$}
     \label{eqn:7.25.tr}
\end{equation} 
a $J(\rho)=\int_{\varphi_\rho}f$. Potom je $J$ konstantní na $(r,\,R)$.
\end{lemma}

\begin{proof}
Bez újmy na obecnosti nechť $z_0=0$. Nechť $\rho \in (r,\,R)$. Potom máme 
$$J(\rho)=i\int_0^{2\pi}f(\rho e^{it})\rho e^{it} \diff t= \\ i\int_0^{2\pi}g(\rho e^{it})\diff t,$$ 
kde $g(z):=f(z)\cdot z$, $z \in P:=P(0,\,r,\,R)$. Dále 
\begin{equation}
    J'(\rho){=} \frac{i}{\rho}\int_0^{2\pi}g'(\rho e^{it})\rho e^{it} \diff t=\frac{1}{\rho}\int_{\varphi_\rho}g'=0, 
    \tag{$\times$}
    \label{eqn:7.25.tim}
\end{equation}
protože $g'$ má PF $g$ na $P$. Platí \cref{eqn:7.25.tim}, protože 
$$\frac{\diff}{\diff \rho}\left(g\left(\rho e^{it}\right)\right)=\frac{\diff g}{\diff x} \cos t+ \frac{\diff g}{\diff y} \sin t\stackrel{\text{CR věta}}{=}g' \cos t+ i g' \sin t=g'\left(\rho e^{it}\right)e^{it}.$$
\end{proof}

\begin{theorem}[Cauchyho vzorec na mezikruží]
Nechť $f \in \mathcal{H}(P)$, kde $P:=P(z_0,\,r,\,R)$. Nechť $r<r_0<R_0<R$ a $s \in P(z_0,\,r_0,\,R_0)$. Potom platí
\begin{equation}
    f(s)=\frac{1}{2 \pi i}\int_{\varphi_{R_0}}\frac{f(z)}{z-s} \diff z - \frac{1}{2 \pi i}\int_{\varphi_{r_0}}\frac{f(z)}{z-s} \diff z\text{,}
    \tag{$\square$}
    \label{eqn:7.26.sq}
\end{equation}
kde $\varphi_\rho$ je jako v \cref{eqn:7.25.tr}.
\end{theorem}

\begin{proof}
Pro $z \in P$ položme 
\begin{equation*}
    h(z)=\twopartdef{\frac{f(z)-f(s)}{z-s}}{z \neq s,}{f'(s)}{z=s.}
\end{equation*}
Potom $h \in \mathcal{H}(P)$, protože $h$ má \uv{odstraněnou} singularitu v $s$. Podle lemmatu máme
$$
\int_{\varphi_{R_0}}h=\int_{\varphi_{R_0}}\frac{f(z) \diff z}{z-s}-f(s) \int_{\varphi_{R_0}}\frac{\diff z}{z-s}\text{, kde poslední integrál je roven }2\pi i \cdot \text{ind}_{\varphi_{R_0}}s=2 \pi i\text{,}
$$
$$
\int_{\varphi_{r_0}}h=\int_{\varphi_{r_0}}\frac{f(z) \diff z}{z-s}-f(s) \int_{\varphi_{r_0}}\frac{\diff z}{z-s}\text{, kde poslední integrál je roven }2\pi i \cdot \text{ind}_{\varphi_{r_0}}s=0{.}
$$
Dále $\int_{\varphi_{R_0}}h=\int_{\varphi_{r_0}}h$, tudíž platí \cref{eqn:7.26.sq}.
\end{proof}

\begin{theorem}[O Laurentově rozvoji holomorfní funkce na mezikruží]
Nechť $P:= \\ P(z_0\,,r,\,R)$, kde $0\leq r < R \leq + \infty$. Nechť $f \in \mathcal{H}(P)$. Potom existuje jediná Laurentova řada 
\begin{equation}
    \sum\limits_{n=-\infty}^{\infty} a_n(z-z_0)^n,  
    \tag{L}
    \label{eqn:7.27.lau}
\end{equation}
která má na $P$ součet $f$.
\end{theorem}

\begin{proof}

\begin{enumerate}
 \item jednoznačnost: Nechť platí $f(z)=\sum\limits_{n=-\infty}^\infty a_n(z-z_0)^n$, $z \in P$. Je-li $\rho \in (r,\,R)$ a $m \in \mathbb{Z}$, pak 
$$
\int_{\varphi_\rho}f(z)(z-z_0)^{-(m+1)} \diff z = \int_{\varphi_\rho} \sum\limits_{n=-\infty}^{\infty} a_n(z-z_0)^{n-m-1} \diff z = \sum\limits_{n=-\infty}^{\infty} a_n \int_{\varphi_\rho}(z-z_0)^{n-m-1} \diff z = 
$$
$=2\pi i \cdot a_m$, kde v druhé rovnosti suma konverguje stejnoměrně na $\langle\varphi_\rho \rangle$ a poslední integrál je roven $0$ pro $n \neq m$ a $2\pi i \cdot \text{ind}_{\varphi_\rho}z_0=2\pi i$ pro $n=m$.

Závěr: koeficienty \cref{eqn:7.27.lau} se dají vyjádřit pomocí součtu $f$ jako 
\begin{equation}
    a_m=\frac{1}{2 \pi i}\int_{\varphi_\rho}\frac{f(z)}{(z-z_0)^{m+1}} dz\text{, } m \in \mathbb{Z}, 
    \tag{**}
    \label{eqn:7.27.**}
\end{equation}
kde $\varphi_\rho$ je jako v \cref{eqn:7.25.tr}. Podle lemmatu integrandy nezávisejí na $\rho \in (r,\,R)$.
\item existence: Nechť $s \in P$. Volme $r<r_0<R_0<R$, aby $s \in P(z_0,\,r_0,\,R_0)$. Potom z Cauchyho vzorce máme \begin{equation}
    f(s)=\frac{1}{2 \pi i}\int_{\varphi_{R_0}}\frac{f(z) \diff z}{z-s}-\frac{1}{2 \pi i}\int_{\varphi_{r_0}}\frac{f(z) \diff z}{z-s}\text{,}
    \tag{a}
    \label{eqn:7.27.a}
\end{equation}
\begin{equation}
    \frac{1}{z-s}=\frac{1}{(z-z_0)-(s-z_0)}=\frac{1}{z-z_0}\cdot \frac{1}{1-\frac{s-z_0}{z-z_0}}=\sum\limits_{n=0}^{+\infty}\frac{(s-z_0)^n}{(z-z_0)^{n+1}}\text{,}
    \tag{b}
    \label{eqn:7.27.b}
\end{equation}
neboť $\left|\frac{s-z_0}{z-z_0}\right|<1$, tedy řada konverguje stejnoměrně pro $z \in \langle\varphi_{R_0} \rangle$;
\begin{equation}
    \frac{1}{z-s}=\frac{1}{(z-z_0)-(s-z_0)}=\frac{(-1)}{s-z_0}\cdot \frac{1}{1-\frac{z-z_0}{s-z_0}}=-\sum\limits_{n=0}^{+\infty}\frac{(z-z_0)^n}{(s-z_0)^{n+1}}\text{,}
    \tag{c}
    \label{eqn:7.27.c}   
\end{equation}
neboť $\left|\frac{z-z_0}{s-z_0}\right|<1$, tedy řada konverguje stejnoměrně pro $z \in \langle\varphi_{r_0} \rangle$.
Dosadíme \cref{eqn:7.27.b}, \cref{eqn:7.27.c} do \cref{eqn:7.27.a} a dostaneme
\begin{equation}
    \begin{aligned}
        f(s)&=\frac{1}{2 \pi i}\int_{\varphi_{R_0}}\sum\limits_{n=0}^{+\infty}\frac{(s-z_0)^n}{(z-z_0)^{n+1}}f(z) \diff z+\frac{1}{2 \pi i}\int_{\varphi_{r_0}}\sum\limits_{n=0}^{+\infty}\frac{(z-z_0)^n}{(s-z_0)^{n+1}}f(z) \diff z \\
        &=\sum\limits_{n=0}^{+\infty}(s-z_0)^n\cdot a_n+\sum\limits_{n=0}^{+\infty}(s-z_0)^{-n-1}\cdot a_{-(n+1)},    
    \end{aligned}
\end{equation}
 kde $a_n$ jsou jako v \cref{eqn:7.27.**}.
\end{enumerate}
\end{proof}

\subsection{Izolované singularity 2}
\begin{theorem}[O Laurentově rozvoji kolem izolované singularity]
Nechť $f \in \mathcal{H}(P(z_0,\,r))$ a $f(z)=\sum\limits_{-\infty}^{+\infty}a_n(z-z_0)^n$, $z \in P(z_0,\,r)$. Potom
\begin{enumerate}
    \item $f$ má v $z_0$ odstranitelnou singularitu $\Leftrightarrow \forall n<0:\;a_n=0$; 
    \item $f$ má v $z_0$ pól násobnosti $p \in \mathbb{N}\Leftrightarrow a_{-p} \neq 0$ a $\forall n<-p: \;a_n=0$; 
    \item $f$ má v $z_0$ podstatnou singularitu $\Leftrightarrow a_n \neq 0$ pro nekonečně mnoho $n<0$.
\end{enumerate}
\end{theorem}

\begin{proof}
\mbox{}
\begin{enumerate}
    \item jasné
    \item $f$ má v $z_0$ pól násobnosti $p$, právě když $g(z):=(z-z_0)^pf(z)$ má v $z_0$ odstranitelnou singularitu a po jejím odstranění je $g(z_0) \neq 0$. Neboli $(z-z_0)^pf(z)=\sum\limits_{n=0}^{+\infty}b_n(z-z_0)^n$, $z \in P(z_0,r)$ a $b_0=g(z_0) \neq 0$, tzn. 
    $$
    f(z)=\frac{b_0}{(z-z_0)^p}+\frac{b_1}{(z-z_0)^{p-1}}+ \dots=\sum\limits_{n=0}^{+\infty}b_n(z-z_0)^{n-p} \text{, } z \in P(z_0,\,r)\text{.}
    $$
    \item Z 1., 2. máme, že $f$ nemá v $z_0$ podstatnou singularitu, právě když $a_n \neq 0$ pro konečně mnoho $n<0$.
\end{enumerate}
\end{proof}

\begin{theorem}[Rozklad holomorfní funkce s konečně mnoha izolovanými singularitami]
Nechť $G \subset \mathbb{C}$ je otevřená, $M \subset G$ je konečná a $f \in \mathcal{H}(G \setminus M)$. Pro každé $s \in M$ označme $H_s$ součet hlavní části Laurentova rozvoje funkce $f$ kolem $s$. Potom existuje jediná $h \in \mathcal{H}(G)$ tak, že $f=\sum\limits_{s\in M} H_s+h$ na $G \setminus M$.
\end{theorem}
\begin{proof}
Zřejmě $\forall s \in M:\ H_s \in \mathcal{H}(\mathbb{C} \setminus \{s \})$. Funkce $h:=f-\sum\limits_{s\in M} H_s$ je holomorfní na $G \setminus M$ a v bodech $s \in M$ má odstranitelné singularity. Skutečně, nechť $s_0 \in M$. Potom existuje $r_0>0$ tak, že $P(s_0,\,r_0) \subset G \setminus M$ a $f=R_{s_0}+H_{s_0}$ na $P(s_0,\,r_0)$, kde $R_{s_0}$ je součet regulární části Laurentova rozvoje $f$ kolem $s_0$ a $R_{s_0}\in \mathcal{H}(U(s_0,\,r_0))$. Tedy na $P(s_0,\,r_0)$ máme
$$h=R_{s_0}+H_{s_0}-\sum\limits_{s\in M} H_s=R_{s_0}-\sum\limits_{
\small{\begin{array}l
s \neq s_0\\
s\in M,
\end{array}}}
H_s \in \mathcal{H}(U(s_0,\,r_0)).$$
\end{proof}

\subsection{Reziduum}
\begin{definition}
Nechť $f \in \mathcal{H}(P(z_0))$ a nechť $f(z)=\sum\limits_{n=-\infty}^{+\infty}a_n(z-z_0)^n$, $z\in P(z_0)$. Potom reziduem $f$ v $z_0$ nazveme číslo res$_{z_0}f:=a_{-1}$.
\end{definition}

\begin{theorem}[Reziduová na hvězdovitých oblastech]
Nechť $G \subset \mathbb{C}$ je hvězdovitá oblast, $M \subset G$ je konečná a $f \in \mathcal{H}(G \setminus M)$. Nechť $\varphi$ je uzavřená křivka v $G \setminus M$. Potom máme 
\begin{equation}
    \int_\varphi f=2\pi i \sum\limits_{s\in M}\text{res}_sf \cdot \text{ind}_\varphi s.
    \tag{RV}
    \label{eqn:7.31.RV}
\end{equation}
\end{theorem}

\begin{note*}
Pro $M=\emptyset$ dostaneme Cauchyho větu.
\end{note*}

\begin{proof}
Podle předchozí věty existuje $h \in \mathcal{H}(G)$ tak, že $f=\sum\limits_{s\in M}H_s+h$ na $G \setminus M$. Potom máme $\int_\varphi f=\sum\limits_{s\in M}\int_\varphi H_s$, protože $\int_\varphi h =0$ z Cauchyho věty pro hvězdovité oblasti. Pro každé $s \in M$:
$$
\int_\varphi H_s(z) \diff z=\int_\varphi \sum\limits_{n=1}^{+\infty}a_{-n}^s\frac{1}{(z-s)^n} \diff z = \sum\limits_{n=1}^{+\infty}a_{-n}^s \int_\varphi \frac{\diff z}{(z-s)^n}=2\pi i \cdot \text{res}_sf\cdot \text{ind}_\varphi s\text{,}
$$
jelikož suma konverguje stejnoměrně na $\langle\varphi \rangle$ a poslední integrál je roven $0$ pro $n \neq 1$ (neboť jinak má integrand PF, a tudíž je integrál přes uzavřenou křivku nulový) a $2 \pi i \cdot \text{ind}_\varphi s$, je-li $n=1$.
\end{proof}
