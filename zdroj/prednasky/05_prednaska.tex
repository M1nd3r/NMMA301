% druhá strana

\begin{theorem}[Cauchyův vzorec pro kruh]
Nechť $G\subset{\Comp  }$ je otevřená a $f\in\Holo (G)$. Nechť $\overline{U(z_0,\, r)}\subset{G}$ a $\varphi{(t)}:=z_0+r.e^{it}$, $t\in{[0,\, 2\pi]}$ $(*)$. Potom platí 
\begin{equation}
    \frac{1}{2\pi{i}}\int_\varphi{\frac{f(z)}{z-s}dz}=
\left\{
	\begin{array}{ll}
		f(s), \hspace{5mm} |s-z_0|<r\\
		0,\hspace{5mm} |s-z_0|>r
	\end{array}
\right.
\tag{$CV_z$}
\label{eqn:cvz}
\end{equation}
\end{theorem}

\begin{proof}
Existuje $R>r$ tak, že $U(z_0,\, R)\subset{G}$.\newline
(i) Nechť $|s-z_0|<r$. Definujme 
\[h(z):=
\left\{
	\begin{array}{ll}
		\frac{f(z)-f(s)}{z-s},\hspace{5mm}z\neq{s}\hspace{2mm} a\hspace{2mm} z\in{G}\\
		f'(s),\hspace{5mm}z=s.
	\end{array}
\right.\]
Potom $h\in\Holo (U(z_0,\, R)\setminus\{s\})$ a spojitá na hvězdovité oblasti $U(z_0,\, R)$. Potom z Cauchyho věty 
\[0=
\frac{1}{2\pi{i}}\int_\varphi{h}=
\frac{1}{2\pi{i}}\int_\varphi{\frac{f(z)}{z-s}}dz-f(s)\cdot \underbrace{\frac{1}{2\pi{i}}\int_\varphi{\frac{dz}{z-s}}}_{=ind_\varphi{s} = 1}
\]
% třetí strana
(ii) Nechť $|s-z_0|>r$. Volme $R'\in(r,|s-z_0|)$, aby $U(z_0,\, R')\subset{G}$. Potom $f(z)/(z-s)$ je holomorfní funkce na $U(z_0,\, R')$ a z Cauchyho věty je 
\[\frac{1}{2\pi{i}}\int_\varphi{\frac{f(z)}{z-s}}dz =0.\]
\end{proof}

\begin{consequence}
Nechť $G\subset{\Comp  }$ je otevřená a $f\in\Holo (G)$. Potom $f$ má komplexní derivaci všech řádů všude na $G$. Nechť $\overline{U(z_0,r)}\subset{G}$ a $\varphi$ je jako v $(*)$. Potom 
\begin{equation}
f^{(k)}(s)=
\frac{k!}{2\pi{i}}\int_\varphi{\frac{f(z)}{(z-s)^{k+1}}}dz,\hspace{5mm} |s-z_0|<r\hspace{2mm} a\hspace{2mm} k = 0,1,2,3,... 
\tag{$CV^{(k)}_z$}
\label{eqn:cvkz}
\end{equation}

Zde $f^{(0)}=f$ a $k$-tá komplexní derivace $f^{(k)}$ je definovaná jako $f^{(k)}=(f^{(k-1)})'$, má-li pravá strana smysl.
\end{consequence}

\begin{proof}
Z věty o derivaci integrálu dle komplexního parametru a \cref{eqn:cvz}, protože 
\[\frac{d^k}{ds^k}\left(\frac{1}{z-s}\right)=
\frac{k!}{(z-s)^{k+1}},\hspace{2mm} z\neq{s}.\]
\end{proof}

% čtvrtá strana

\begin{theorem}[Morera]\label{thm:morera}
Nechť $f$ je spojitá funkce na otevřené $G\subset{\Comp  }$. Potom $f\in\Holo (G)$, právě když
\begin{equation}
    \begin{aligned}
\int_{\partial\triangle}{f} =0 \hspace{3mm} \text{pro každý trojúhelník }\triangle\subset{G}.    
    \end{aligned}
    \label{eqn:morera}
\end{equation}
\end{theorem}

\begin{proof}
"$\Rightarrow$": Goursatovo lemma\newline
"$\Leftarrow$": Nechť $\Ukruh :=U(z_0,\, R)$ je libovolný kruh v $G$. Protože $f$ je spojitá na $\Ukruh $, $\Ukruh $ je hvězdovitá oblast a 
\[\int_{\partial\triangle}{f} =0\]
pro každý trojúhelník $\triangle\subset\Ukruh $, má $f$ na $\Ukruh $ primitivní funkci $F$, to znamená, že $f=F'$ na $\Ukruh $. Protože $F\in\Holo (\Ukruh )$, máme $f'=F''$ na $\Ukruh $, tudíž $f$ je holomorfní na $\Ukruh $. Protože $\Ukruh $ byl libovolný kruh v $G$, je $f\in\Holo (G)$.
\end{proof}

% pátá strana

\begin{theorem}[Cauchyho odhady]
Nechť $z_0\in\Comp  $, $r\in(0,\, +\infty)$ a $f$ je holomorní funkce na otevřené množině obsahující $\overline{U(z_0,\, r)}$. Potom pro každé $k=0,1,2,...$ je 
\begin{equation}
\forall{s\in{\Ukruh :=U(z_0,r)}}:\hspace{5mm}|f^{(k)}(s)|\leq\frac{r\cdot k!}{(d(s))^{k+1}}\cdot \max_{\partial\Ukruh }{|f|},    
\tag{$CO_1$}
\label{eqn:co1}
\end{equation}
kde $d(s) := dist(s,\partial\Ukruh )\overset{def.}{:=}\underset{{z\in{\partial{\Ukruh }}}}{min}{|s-z|}$
\begin{equation}
\forall{s\in{U\left(z_0,\frac{r}{2}\right)}}:\hspace{5mm}|f^{(k)}(s)|\leq\frac{k!\cdot2^{k+1}}{r^k}\cdot\max_{\partial{U}}{|f|}\cdot
\tag{$CO_2$}
\label{eqn:co2}
\end{equation}
\begin{equation}
    |f^{(k)}(z_0)|\leq\frac{k!}{r^k}\cdot\max_{\partial{U}}{|f|}. 
\tag{$CO_3$}
\label{eqn:co3}
\end{equation}
\end{theorem}

\begin{proof}
\cref{eqn:co1} dostaneme z \cref{eqn:cvkz}, protože 
\[|f^{(k)}(s)|=
\left\lvert\frac{k!}{2\pi{i}}\int_\varphi{\frac{f(z)}{(z-s)^{k+1}}}dz\right\lvert\leq
\frac{k!}{2\pi}\cdot 2\pi{r}\cdot\frac{1}{(d(s))^{k+1}}\cdot \max_{\partial\Ukruh }{|f|}\]
a $|z-s|\geq d(s)$, $z\in\partial\Ukruh =\langle\varphi\rangle$, 
zde $\varphi(t)=z_0+r\cdot e^{it}$, $t\in\left[0, 2\pi\right]$.\newline
\cref{eqn:co2} plyne z \cref{eqn:co1}, protože $d(s)\geq\frac{r}{2}$ $\forall{s\in{U(z_0,\, r/2)}}$.\newline
\cref{eqn:co3} plyne z \cref{eqn:co1}, protože $d(z_0)=r$. 
\end{proof}

% šestá strana

\begin{theorem}[Liouville] \label{thm:liouville}
Je-li $f$ holomorfní  a omezená na $\Comp  $, potom je $f$ konstantní.
\end{theorem}

\begin{proof}
Ukážeme, že $f'=0$ na $\Comp$. 
Označme $M:=\sup_{\,\Comp}{|f|}<+\infty$. Nechť $z_0\in\Comp$. Z \cref{eqn:co3} dostaneme pro každé $r>0$
\[|f'(z_0)|\leq
\frac{1}{r}\max_{\partial{U(z_0,r)}}{|f|}\leq
\frac{M}{r}\underset{{r\to{+\infty}}}{\to}0,\]
tudíž $f'(z_0)=0$.
\end{proof}

\begin{consequence}[Základní věta algebry]
V $\Comp  $ má polynom stupně aspoň $1$ vždy aspoň jeden kořen.
\end{consequence}

\begin{proof}
Nechť $p(z)={a_0}{z^n}+{a_1}{z^{n-1}}+...+{a_n}$, kde $a_j\in\Comp  $, $a_0\neq{0}$ a $n\geq{1}$.\newline
Sporem: Předpokládejme, že $p\neq{0}$ na $\Comp  $. Položme $f:=1/p$. Potom $f$ je holomorfní a omezená na $\Comp  $, tudíž dle Liouvilleovy věty je $f$ i $p$ konstantní. Tedy $p'=0$ a $0=p^{(n)}=n!{a_0}$, což je spor.\newline
% sedmá strana
Omezenost $f$: Máme
\[|f(z)|=\left\lvert\frac{1}{z^n.\left(a_0+\frac{a_1}{z}+...+\frac{a_n}{z^n}\right)}\right\lvert\leq
\frac{1}{r^n}.\frac{1}{|a_0|-\frac{|a_1|}{r}-...-\frac{|a_n|}{r^n}}\longrightarrow 0\]
pro $r=|z|\to+\infty$.

Existuje $r_0\in(0,\, +\infty)$ tak, že $|f(z)|\leq{1}$, je-li $|z|>r_0$. Funkce $f$ je omezená na $\overline{U(0,\, r_0)}$, protože je tam spojitá.
\end{proof}

\begin{lemma}
Nechť $\varphi$ je křivka v $\Comp  $, $f_n$ jsou spojité funkce na $\langle\varphi\rangle$ pro $n=1,2,3,...$ a $f_n\rightrightarrows{f}$ na $\langle\varphi\rangle$. Potom $f$ je spojitá na $\langle\varphi\rangle$ a 
\[\int_\varphi{f_n}\longrightarrow\int_\varphi{f}.\]
\end{lemma}

\begin{proof}
Máme
\[0\leq\left\lvert{\int_\varphi{f_n}-\int_\varphi{f}}\, \right\lvert=
\left\lvert{\int_\varphi{(f_n-f)}}\right\lvert\leq{}
V(\varphi)\cdot\max_{\langle\varphi\rangle}{|f_n-f|}\overset{n\to\infty}{\longrightarrow}0.\]
\end{proof}

% osmá strana
\begin{theorem}[Weierstrass]
Nechť $G\subset{\Comp  }$ je otevřená, $f_n\in\Holo (G)$ pro $n\in\N$ a $f_n\overset{loc}{\rightrightarrows}f$ na $G$. Potom $f\in\Holo (G)$ a $f_n^{(k)}\overset{loc}{\rightrightarrows}f^{(k)}$ na $G$ pro každé $k\in\N$.
\end{theorem}

\begin{proof}
\circled{1} Zřejmě je $f$ spojitá. Nechť $\triangle$ je trojúhelník v $G$. Potom 
\[0=\int_{\partial\triangle}{f_n}\overset{Lemma}{\longrightarrow}
\int_{\partial\triangle}{f}=0\]
Z Morerovy věty je $f\in\Holo (G)$.

\circled{2} Nechť $k\in\N$ a $z_0\in{G}$. Volme $r>0$, aby $\overline{U(z_0,r)}\subset{G}$. Potom z $(CO_2)$ máme:
\[\forall{s\in{U\left(z_0,\frac{r}{2}\right)}}:\hspace{3mm}
{\left\lvert{f_n^{(k)}(s)-f^{(k)}(s)}\right\lvert}=
\left\lvert{\left({f_n-f}\right)^{(k)}(s)}\right\lvert\leq
\frac{k!.2^{k+1}}{r^k}.\max_{\partial{U(z_0,r)}}{|f_n-f|}
\overset{n\to+\infty}{\longrightarrow}0\]
\end{proof}

% devátá strana
\section{\texorpdfstring{Mocninné řady}{Mocninné rady}}

\begin{definition}
Nechť $\{a_n\}_{n=0}^\infty\subset\Comp  $ a $z_0\in\Comp  $. Potom 
\begin{equation}
\sum_{n=0}^\infty{a_n\cdot(z-z_0)^n}, \hspace{5mm} z\in\Comp  
\label{eqn:rada}
\end{equation}
je mocninná řada o středu $z_0$ s koeficienty $\{a_n\}_{n=0}^\infty$.
\end{definition}

\begin{properties}
\circled{1} \textbf{Konvergence} (na cvičení) \newline
Existuje jediné $R\in{[0,+\infty]}$ takové, že 
\begin{itemize}
    \item řada \cref{eqn:rada} konverguje absolutně a lokálně stejnoměrně na  $U(z_0,R):=\{z\in\Comp   : |z-z_0|<R\}$, 
    \item řada \cref{eqn:rada} diverguje pro $|z-z_0|>R$. 
\end{itemize}
Číslo $R$ se nazývá poloměr konvergence \cref{eqn:rada} a platí, že
\[R=\frac{1}{
\underset{n\to+\infty}{\limsup}{\sqrt[\leftroot{-3}\uproot{3}n]{|a_n|}}},\]
kde položíme $\frac{1}{0}=+\infty$, $\frac{1}{+\infty}=0$.

\circled{2} Označíme-li součet \cref{eqn:rada} na $U(z_0,\, R)$ jako $f$, potom je $f\in\Holo (U(z_0,\, R))$ a 
\[
\forall{k\in\N_0}\hspace{2mm}\forall{z\in{U(z_0,\,R)}}: \hspace{5mm}
f^{(k)}(z)=
\sum_{n=k}^{+\infty}{a_n.n.(n-1)...(n-k+1)(z-z_0)^{n-k}},\]
speciálně $a_k=\frac{f^{(k)}(z_0)}{k!}$.
\end{properties}

\begin{note}
Mocninnou řadu derivujeme "člen po členu", můžeme na $U(z_0,\,r)$ zaměnit sumu a komplexní derivaci.
\end{note}

\begin{proof}
Užijeme Weierstrassovu větu na 
\[S_n(z):=\sum_{n=0}^N{a_n(z-z_0)^n},\hspace{5mm} z\in{U(z_0,\,R)}\]
Dosadíme-li do \cref{eqn:rada} $z=z_0$, máme $f^{(k)}(z_0)={a_k}.{k!}$
\end{proof}